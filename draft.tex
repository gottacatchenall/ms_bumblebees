%!TEX TS-program = xelatex
\documentclass[11pt]{article}

\usepackage[english]{babel}

\usepackage{amsmath,amssymb,amsfonts}
\usepackage[utf8]{inputenc}
\usepackage[T1]{fontenc}
\usepackage{stix2}
\usepackage[scaled]{helvet}
\usepackage[scaled]{inconsolata}

\usepackage{lastpage}

\usepackage{setspace}

\usepackage{ccicons}

\usepackage[hang,flushmargin]{footmisc}

\usepackage{geometry}

\setlength{\parindent}{0pt}
\setlength{\parskip}{6pt plus 2pt minus 1pt}

\usepackage{fancyhdr}
\renewcommand{\headrulewidth}{0pt}\providecommand{\tightlist}{%
  \setlength{\itemsep}{0pt}\setlength{\parskip}{0pt}}

\makeatletter
\newcounter{tableno}
\newenvironment{tablenos:no-prefix-table-caption}{
  \caption@ifcompatibility{}{
    \let\oldthetable\thetable
    \let\oldtheHtable\theHtable
    \renewcommand{\thetable}{tableno:\thetableno}
    \renewcommand{\theHtable}{tableno:\thetableno}
    \stepcounter{tableno}
    \captionsetup{labelformat=empty}
  }
}{
  \caption@ifcompatibility{}{
    \captionsetup{labelformat=default}
    \let\thetable\oldthetable
    \let\theHtable\oldtheHtable
    \addtocounter{table}{-1}
  }
}
\makeatother

\usepackage{array}
\newcommand{\PreserveBackslash}[1]{\let\temp=\\#1\let\\=\temp}
\let\PBS=\PreserveBackslash

\usepackage[breaklinks=true]{hyperref}
\hypersetup{colorlinks,%
citecolor=blue,%
filecolor=blue,%
linkcolor=blue,%
urlcolor=blue}
\usepackage{url}

\usepackage{caption}
\setcounter{secnumdepth}{0}
\usepackage{cleveref}

\usepackage{graphicx}
\makeatletter
\def\maxwidth{\ifdim\Gin@nat@width>\linewidth\linewidth
\else\Gin@nat@width\fi}
\makeatother
\let\Oldincludegraphics\includegraphics
\renewcommand{\includegraphics}[1]{\Oldincludegraphics[width=\maxwidth]{#1}}

\usepackage{longtable}
\usepackage{booktabs}

\usepackage{color}
\usepackage{fancyvrb}
\newcommand{\VerbBar}{|}
\newcommand{\VERB}{\Verb[commandchars=\\\{\}]}
\DefineVerbatimEnvironment{Highlighting}{Verbatim}{commandchars=\\\{\}}
% Add ',fontsize=\small' for more characters per line
\usepackage{framed}
\definecolor{shadecolor}{RGB}{248,248,248}
\newenvironment{Shaded}{\begin{snugshade}}{\end{snugshade}}
\newcommand{\KeywordTok}[1]{\textcolor[rgb]{0.13,0.29,0.53}{\textbf{#1}}}
\newcommand{\DataTypeTok}[1]{\textcolor[rgb]{0.13,0.29,0.53}{#1}}
\newcommand{\DecValTok}[1]{\textcolor[rgb]{0.00,0.00,0.81}{#1}}
\newcommand{\BaseNTok}[1]{\textcolor[rgb]{0.00,0.00,0.81}{#1}}
\newcommand{\FloatTok}[1]{\textcolor[rgb]{0.00,0.00,0.81}{#1}}
\newcommand{\ConstantTok}[1]{\textcolor[rgb]{0.00,0.00,0.00}{#1}}
\newcommand{\CharTok}[1]{\textcolor[rgb]{0.31,0.60,0.02}{#1}}
\newcommand{\SpecialCharTok}[1]{\textcolor[rgb]{0.00,0.00,0.00}{#1}}
\newcommand{\StringTok}[1]{\textcolor[rgb]{0.31,0.60,0.02}{#1}}
\newcommand{\VerbatimStringTok}[1]{\textcolor[rgb]{0.31,0.60,0.02}{#1}}
\newcommand{\SpecialStringTok}[1]{\textcolor[rgb]{0.31,0.60,0.02}{#1}}
\newcommand{\ImportTok}[1]{#1}
\newcommand{\CommentTok}[1]{\textcolor[rgb]{0.56,0.35,0.01}{\textit{#1}}}
\newcommand{\DocumentationTok}[1]{\textcolor[rgb]{0.56,0.35,0.01}{\textbf{\textit{#1}}}}
\newcommand{\AnnotationTok}[1]{\textcolor[rgb]{0.56,0.35,0.01}{\textbf{\textit{#1}}}}
\newcommand{\CommentVarTok}[1]{\textcolor[rgb]{0.56,0.35,0.01}{\textbf{\textit{#1}}}}
\newcommand{\OtherTok}[1]{\textcolor[rgb]{0.56,0.35,0.01}{#1}}
\newcommand{\FunctionTok}[1]{\textcolor[rgb]{0.00,0.00,0.00}{#1}}
\newcommand{\VariableTok}[1]{\textcolor[rgb]{0.00,0.00,0.00}{#1}}
\newcommand{\ControlFlowTok}[1]{\textcolor[rgb]{0.13,0.29,0.53}{\textbf{#1}}}
\newcommand{\OperatorTok}[1]{\textcolor[rgb]{0.81,0.36,0.00}{\textbf{#1}}}
\newcommand{\BuiltInTok}[1]{#1}
\newcommand{\ExtensionTok}[1]{#1}
\newcommand{\PreprocessorTok}[1]{\textcolor[rgb]{0.56,0.35,0.01}{\textit{#1}}}
\newcommand{\AttributeTok}[1]{\textcolor[rgb]{0.77,0.63,0.00}{#1}}
\newcommand{\RegionMarkerTok}[1]{#1}
\newcommand{\InformationTok}[1]{\textcolor[rgb]{0.56,0.35,0.01}{\textbf{\textit{#1}}}}
\newcommand{\WarningTok}[1]{\textcolor[rgb]{0.56,0.35,0.01}{\textbf{\textit{#1}}}}
\newcommand{\AlertTok}[1]{\textcolor[rgb]{0.94,0.16,0.16}{#1}}
\newcommand{\ErrorTok}[1]{\textcolor[rgb]{0.64,0.00,0.00}{\textbf{#1}}}
\newcommand{\NormalTok}[1]{#1}

\newlength{\cslhangindent}
\setlength{\cslhangindent}{1.5em}
\newlength{\csllabelwidth}
\setlength{\csllabelwidth}{3em}
\newenvironment{CSLReferences}[3] % #1 hanging-ident, #2 entry spacing
 {% don't indent paragraphs
  \setlength{\parindent}{0pt}
  % turn on hanging indent if param 1 is 1
  \ifodd #1 \everypar{\setlength{\hangindent}{\cslhangindent}}\ignorespaces\fi
  % set entry spacing
  \ifnum #2 > 0
  \setlength{\parskip}{#2\baselineskip}
  \fi
 }%
 {}
\usepackage{calc} % for \widthof, \maxof
\newcommand{\CSLBlock}[1]{#1\hfill\break}
\newcommand{\CSLLeftMargin}[1]{\parbox[t]{\maxof{\widthof{#1}}{\csllabelwidth}}{#1}}
\newcommand{\CSLRightInline}[1]{\parbox[t]{\linewidth}{#1}}
\newcommand{\CSLIndent}[1]{\hspace{\cslhangindent}#1}\geometry{verbose,letterpaper,tmargin=2.2cm,bmargin=2.2cm,lmargin=2.2cm,rmargin=2.2cm}

\usepackage{lineno}
\usepackage[nolists,noheads]{endfloat}

\pagestyle{plain}

\tolerance=1
\emergencystretch=\maxdimen
\hyphenpenalty=10000
\hbadness=10000

\doublespacing

\fancypagestyle{normal}
{
  \fancyhf{}
  \fancyfoot[R]{\footnotesize\sffamily\thepage\ of \pageref*{LastPage}}
}
\begin{document}
\raggedright
\thispagestyle{empty}
{\Large\bfseries\sffamily Building a better metaweb: predicting
spatiotemporally explicit plant-pollinator networks}
\vskip 5em

%
\href{https://orcid.org/0000-0002-6506-6487}{Michael D.\,Catchen}%
%
\,\textsuperscript{1,2}\quad %
Paul\,CaraDonna%
%
\,\textsuperscript{3,4}\quad %
Jane E.\,Ogilvie%
%
\,\textsuperscript{3}\quad %
\href{https://orcid.org/0000-0001-9051-0597}{Francis\,Banville}%
%
\,\textsuperscript{3}\quad %
\href{https://orcid.org/0000-0002-2151-6693}{Dominique\,Caron}%
%
\,\textsuperscript{1,2}\quad %
\href{https://orcid.org/0000-0002-6248-3007}{Philippe\,Desjardins-Proulx}%
%
\,\textsuperscript{5,2}\quad %
\href{https://orcid.org/0000-0001-9019-0108}{Norma R.\,Forero-Muñoz}%
%
\,\textsuperscript{5,2}\quad %
\href{https://orcid.org/0000-0002-4498-7076}{Dominique\,Gravel}%
%
\,\textsuperscript{6,2}\quad %
\href{https://orcid.org/0000-0002-6004-4027}{Laura\,Pollock}%
%
\,\textsuperscript{1,2}\quad %
\href{https://orcid.org/0000-0001-6067-1349}{Tanya\,Strydom}%
%
\,\textsuperscript{5,2}\quad %
\href{https://orcid.org/0000-0002-0735-5184}{Timothée\,Poisot}%
%
\,\textsuperscript{5,2}\quad %
Julian\,Resasco%
%
\,\textsuperscript{7}\quad %
\href{https://orcid.org/0000-0001-6075-8081}{Andrew\,Gonzalez}%
%
\,\textsuperscript{1,2}

\textsuperscript{1}\,McGill University\quad \textsuperscript{2}\,Québec
Centre for Biodiversity Sciences\quad \textsuperscript{3}\,Rocky
Mountain Biological Laboratory\quad \textsuperscript{4}\,Chicago
Bontanic Garden\quad \textsuperscript{5}\,Université de
Montréal\quad \textsuperscript{6}\,Université de
Sherbrooke\quad \textsuperscript{7}\,University of Colorado Boulder


\textbf{Correspondance to:}\\
Michael D. Catchen --- \texttt{michael.catchen@mail.mcgill.ca}\\

\vfill
This work is released by its authors under a CC-BY 4.0 license\hfill\ccby\\
Last revision: \emph{\today}

\clearpage
\thispagestyle{empty}

\vfill


        {\bfseries Purpose:}\,This template provides a series of scripts
to render a markdown document into an interactive website and a series
of PDFs.\\%
        {\bfseries Motivation:}\,It makes collaborating on text with
GitHub easier, and means that we never need to think about the
output.\\%
        {\bfseries Internals:}\,GitHub actions and a series of python
scritpts. The markdown is handled with \texttt{pandoc}.\\%
    
\vfill

\clearpage
\linenumbers
\pagestyle{normal}

\hypertarget{abstract}{%
\subsection{Abstract}\label{abstract}}

Using a data set of {[}DESCRIBE EACH DATASET IN A NICE WAY{]}, we
predict a spatiotemporally explicit metaweb of interactions between
bumblebees (\emph{Bombus}) and wildflowers (within \emph{find clade}).
We integrate this data with crowdsourced occurrence data and climate
data to {[}best paint the picture of the Colorado bumblebee-plant
metaweb{]}. Using temporal climate data, we forecast how the
spatiotemporal overlap of interacting species will change under proposed
climate scenarios. We use this to estimate what interactions between
bees and plants need the most attention to prevent the spatiotemporal
decoupling of an interactions from threatening ecosystem functioning or
the persistence of a species.

\hypertarget{introduction}{%
\section{Introduction}\label{introduction}}

Species interactions are important. It is ultimately interactions
between individuals of different species that drive the structure,
dynamics, and persistence of ecosystems, and the abundance and diversity
of the species within them. Plant-pollinator interactions specifically
drive the function and persistence of ``architecture of biodiversity''
(Bascompte \& Jordano 2007). However, we are far from a robust
understanding of plant-pollinator networks. This is because sampling
interactions is costly. Interactions vary in space and time (Poisot
\emph{et al.} 2015)---particularlly relevent in this system (CaraDonna
\emph{et al.} 2014). This is why there is interest in using models to
predict interactions from sparse data (\textbf{Strydom2021?}). In this
paper, we combine several datasets, each spanning several years, to
produce spatially and temporally explicit predictions of the bumblebee
(genus \emph{Bombus}) and wildflower pollination network across the
state of Colorado.

We do this in two parts: (1) metaweb prediction and (2) conditioning our
metaweb prediction on co-occurrence probability. First, we build a model
to predict the metaweb---the network of \emph{all} interactions,
aggregated across all times and spatial locations---of \emph{Bombus} and
wildflower species across Colorado. (Why do this? The metaweb is more
predictable than local interactions.) We do this using network embedding
(\textbf{cite?}). Network embedding takes each node in the network
(either a bumblebee or a wildflower) and represents it in a latent \(n\)
dimensional space. Combination of running models on Temporal niche (T),
Phylogenetic niche (P), Environmental niche (E), and relative abundance
in community (RA).

Second, we then use this metaweb to predict the structure of networks at
specific locations and times of year (Gravel \emph{et al.} 2019).
Finally we suggest a map of sampling priority, which suggests the
locations to sample that will best improve our understanding of the
Colorado \emph{Bombus} pollination metaweb.

Why is this good for science, what does this contribute to our
understanding of plant-pollinator ints, networks, Bombus, predictive
models, etc., and how can these results be useful.

\hypertarget{data}{%
\section{Data}\label{data}}

We use three separate field datasets to estimate the Colorado
\emph{Bombus} metaweb.

\hypertarget{methods}{%
\section{Methods}\label{methods}}

\begin{figure}
\centering
\includegraphics{./figures/concept.png}
\caption{todo}
\end{figure}

\hypertarget{predicting-the-metaweb}{%
\section{Predicting the metaweb}\label{predicting-the-metaweb}}

\hypertarget{feature-embedding}{%
\subsection{Feature Embedding}\label{feature-embedding}}

\hypertarget{environmental-niche-features}{%
\subsubsection{Environmental niche
features}\label{environmental-niche-features}}

We take the 19 BioClim layers from CHELSA (cite; 1km resolution) and a
map of elevation and PCA them. A resulting 4 layers cover 99.5\% of the
variance. We use species occurrence data from GBIF, and consider each
occurrence record as a point in environment space. Then we fit a
multivariate normal distribution to these points in environmental space.

\hypertarget{temporal-niche-features}{%
\subsubsection{Temporal niche features}\label{temporal-niche-features}}

We take the mean and variance of the distribution of number of
observations per week of year in the interaction field data.

\hypertarget{phylogenetic-features}{%
\subsubsection{Phylogenetic features}\label{phylogenetic-features}}

\hypertarget{phylogeny-construction}{%
\paragraph{Phylogeny Construction}\label{phylogeny-construction}}

We construct phylogenies for both \emph{Bombus} and wildflower species
using barcode markers, mitochondrial COI and chloroplast rbcL,
respectively. These sequences were obtained from NCBI GenBank for all
species. For species for which no sequence was available (only a handful
of plants), their was substituted with a barcode from a member of the
same genus. Justify why this is fine here.

These sequences were aligned using ClustalOmega v???, and then a
posterior distribution of phylogenies and consensus tree was obtained
via MrBayes v??, using XX substition model with gamma-distributed rates.
Run until convergence, which here we define as the standard-deviation of
splits falling below 0.1.

\hypertarget{creating-an-embedding-from-phylogenies}{%
\paragraph{Creating an embedding from
phylogenies}\label{creating-an-embedding-from-phylogenies}}

We simulate traits. Relationship between number of traits, num output
PCA dimensions, and number of used dimensions in the model matter.
Describe how that works.

\hypertarget{relative-abundance}{%
\subsubsection{Relative Abundance}\label{relative-abundance}}

This embedding is the simplest.

\hypertarget{metaweb-model-fitting-and-validation}{%
\subsection{Metaweb Model Fitting and
Validation}\label{metaweb-model-fitting-and-validation}}

We fit a bunch of models using MLJ.jl.

Some of them are bagged, some are not bagged.

AUC-ROC and AUC-PR values below in fig.~\ref{fig:prroc}

\begin{figure}
\hypertarget{fig:prroc}{%
\centering
\includegraphics{./figures/PR_ROC.png}
\caption{todo}\label{fig:prroc}
}
\end{figure}

What does this tell us? The ensemble model is regularly the best for
ROC, but not for PR. This illustrates an inherent trade-off between
models being as ``right'' as possible versus a model being useful for
discovering false-negatives.

\hypertarget{predicting-networks-in-space-and-time}{%
\section{Predicting networks in space and
time}\label{predicting-networks-in-space-and-time}}

Now that we have a metaweb, we can extend this to predict interactions
at particular places and times by decomposing the probability of
interaction at particular place and time into probability of interaction
multiplied by probability of co-occurrence via properties of conditional
probability (Gravel \emph{et al.} 2019).

How do we define? \(P(i \leftrightarrow j)\) Is it
\(P(A_{ij})P(O_{ij})\) or \(P(A_{ij})P(O_i)P(O_j)\)

\textbf{\emph{Figure 3: Maps over time figure and Prob(Connectance)
vs.~Month figure}}

\hypertarget{prioritizing-spatial-sampling-of-pollinator-interactions}{%
\section{Prioritizing spatial sampling of pollinator
interactions}\label{prioritizing-spatial-sampling-of-pollinator-interactions}}

How do we improve out understanding of this pollination network, or
determine if it is changing over time?

\textbf{\emph{Figure 4: Uncertainty and sampling priority map}}

\hypertarget{discussion}{%
\section{Discussion}\label{discussion}}

We predict things alright. Emphasizes how heterogenous data sources can
improve interaction prediction (rocpr fig). Advances network embedding
as a framework for prediction of species interaction networks.

\hypertarget{refs}{}
\begin{CSLReferences}{1}{0}
\leavevmode\hypertarget{ref-Bascompte2007PlaMut}{}%
Bascompte, J. \& Jordano, P. (2007). Plant-Animal Mutualistic Networks:
The Architecture of Biodiversity. \emph{Annual Review of Ecology,
Evolution, and Systematics}, 38, 567--593.

\leavevmode\hypertarget{ref-CaraDonna2014ShiFlo}{}%
CaraDonna, P.J., Iler, A.M. \& Inouye, D.W. (2014). Shifts in flowering
phenology reshape a subalpine plant community. \emph{Proceedings of the
National Academy of Sciences}, 111, 4916--4921.

\leavevmode\hypertarget{ref-Gravel2019BriElt}{}%
Gravel, D., Baiser, B., Dunne, J.A., Kopelke, J.-P., Martinez, N.D.,
Nyman, T., \emph{et al.} (2019). Bringing Elton and Grinnell together: A
quantitative framework to represent the biogeography of ecological
interaction networks. \emph{Ecography}, 42, 401--415.

\leavevmode\hypertarget{ref-Poisot2015SpeWhy}{}%
Poisot, T., Stouffer, D.B. \& Gravel, D. (2015). Beyond species: Why
ecological interaction networks vary through space and time.
\emph{Oikos}, 124, 243--251.

\end{CSLReferences}

\end{document}
